\documentclass[12pt]{article}
\usepackage[UTF8]{ctex}
\usepackage{lingmacros}
\usepackage{tree-dvips}
\begin{document}

\section*{Introduction to Thread, Process, and Parallel}



{\small1. 我们知道计算机的核心是CPU,它就像是一座工厂。}\\
{\small2. 但工厂的电力有限,每次只能给一个车间使用。也就是说当,一个车间开工,其他车间就必须停工。这就对应着“每个CPU每次只能支持一个任务运行”。}\\
{\small3. Process(进程)就像是工厂的车间,它代表了CPU所能处理的单个任务。任意时刻,CPU总是运行一个进程,其他进程处于停工状态。}\\
{\small4. 每个车间(Process)里可以有多个工人,线程(Thread)就像是车间里的工人,一个进程(Process)中可以包含多个线程(Thread),多个线程之间可以相互协作,共同完成一个任务。}\\
{\small5. 车间(Process)的空间是工人们(Thread)共享的,就好比车间内有很多房间都是允许工人们随意进出的。
这里提到的房间就好比计算机的内存空间,意味着一个进程(Process)的内存空间共享的,每个线程(Thread)
都可以共享这些内存。}\\
{\small6. 但是每个内存的大小有所不同,有些内存可能只能同时容纳一个线程(Thread)。当有多个线程都想要
使用这样的内存时,其他线程(Thread)必须等到当前线程使用完该内存后,才能占用这一块内存。}
{\small7. 从上面一步能发现,很有可能出现多个线程(Thread)想要同时占用同一个内存空间的情况。为了发生混乱,
我们可以在门(内存空间)上加一把锁。先占用的Thread对内存空间上锁,从而让之后的Thread进行排队。
这个锁就叫“互斥锁”(Mutual Exclusion, i.e. Mutex)。}\\
{\small8. 但并不是每个房间(内存空间)都是只能容纳一个Thread的。有些内存空间可以同时容纳 n 个Thread。
同理,为了防止超过 n 个Thread同时占用这个内存空间,我们同样可以加 n 个Mutex(互斥锁)。}\\
{\small9. 这时候,除了加 n 个Mutex外,我们还要在门口加 n 个钥匙。每进去一个人(Thread)就取一把钥匙,
出来时再把钥匙挂回原处。如果门口的钥匙架空了,就需要排队。这种方法叫做“信号量”(Semaphore),
用来保证多个线程(Thread)之间的工作不会相互冲突。}\\
{\small10. 很容易发现,Mutex 其实是 Semaphore的一种特殊情况(n=1)。也就是说,理论上可以用后者来代替
前者。但由于Mutex较为简单,且实现起来效率高,所以在必须保证资源独占的情况下,还是采用Mutex的设计。}\\
{\small11. 综上,操作系统的设计可以归结为三点:\\
(1). 以多进程(Process)形式,允许多个任务同时运行;\\
(2). 以多线程(Thread)形式,允许单个任务分成不同的部分运行;\\
(3). 提供协调机制,一方面防止进程(Process)之间和线程(Thread)之间产生冲突,另一方面允许
    进程之间和线程之间贡献资源。
是能够被scheduler进行独立管理的最小序列}\\

\subsection*{How to handle topicalization}

I'll just assume a tree structure like (\ex{1}).

{\small
\enumsentence{Structure of A$'$ Projections:\\ [2ex]
\begin{tabular}[t]{cccc}
    & \node{i}{CP}\\ [2ex]
    \node{ii}{Spec} &   &\node{iii}{C$'$}\\ [2ex]
        &\node{iv}{C} & & \node{v}{SAgrP}
\end{tabular}
\nodeconnect{i}{ii}
\nodeconnect{i}{iii}
\nodeconnect{iii}{iv}
\nodeconnect{iii}{v}
}
}

\subsection*{Mood}

Mood changes when there is a topic, as well as when
there is WH-movement.  \emph{Irrealis} is the mood when
there is a non-subject topic or WH-phrase in Comp.
\emph{Realis} is the mood when there is a subject topic
or WH-phrase.

\begin{thebibliography}
  \\\texttt{https://www.ruanyifeng.com/blog/2013/04/processes\_and\_threads.html}
\end{thebibliography}
  

\end{document}